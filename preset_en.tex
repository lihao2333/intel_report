%==================== 数学符号公式 ============
\usepackage{amsmath}                 % AMS LaTeX宏包
\usepackage[style=1]{mdframed}
\usepackage{amsthm}
\usepackage{pdfpages}
\usepackage{amsfonts}
\usepackage{mathrsfs}                % 英文花体字 体
\usepackage{bm}                      % 数学公式中的黑斜体
\usepackage{manfnt}           % 一些图标,如 \dbend
\usepackage{lettrine}                % 首字下沉,命令\lettrine
\def\attention{\lettrine[lines=2,lraise=0,nindent=0em]{\large\textdbend\hspace{1mm}}{}}
\usepackage{longtable}
\usepackage[toc,page]{appendix}
\usepackage{geometry}                % 页边距调整
\geometry{top=3.0cm,bottom=2.7cm,left=2.5cm,right=2.5cm}

%====================清单==========================
\usepackage{enumitem}
\setlist[description]{leftmargin=\parindent,labelindent=\parindent}
\setlist[enumerate]{leftmargin=1.5\parindent,labelindent=\parindent}
%====================公式按章编号==========================
\numberwithin{equation}{section}
\numberwithin{table}{section}
\numberwithin{figure}{section}
%================= 基本格式预置 ===========================
\usepackage{fancyhdr}
\pagestyle{fancy}
\fancyhf{}  
\fancyhead[C]{\zihao{5}  \kaishu Ainascere Gallery--Fast style transformation system}
\fancyfoot[C]{~\zihao{5} \thepage~}
\renewcommand{\headrulewidth}{0.65pt} 
\CTEXsetup[format={\centering\bfseries\zihao{-2}},name={Chapter\;,}]{section}
\CTEXsetup[nameformat={\bfseries\zihao{3}}]{subsection}
\CTEXsetup[nameformat={\bfseries\zihao{4}}]{subsubsection}

%================== 表格 =========================
\usepackage{makecell,rotating,multirow,diagbox}
\newcommand{\addtbl}[1]
{
  \input{table/#1}
}
%==================  图片 =========================
\newcommand{\addfig}[4][0.5]
{
  \begin{figure}[H]
  \centering\includegraphics[width=#1\linewidth]{figure/#2}
  \caption{#4}
  \label{#3}
  \end{figure}
}
%================== 图形支持宏包 =========================
\usepackage{subfigure}
\usepackage{graphicx}                % 嵌入png图像
\usepackage{color,xcolor}            % 支持彩色文本、底色、文本框等
\usepackage{hyperref}                % 交叉引用
%\usepackage{caption}
\usepackage[font=small,labelfont=bf,labelsep=space]{caption}
\captionsetup{figurewithin=section}
%==================== 源码和流程图 =====================
\newcommand{\addcode}[2][None]
{
	\lstinputlisting[language=#1]{code/#2}
}

\usepackage{listings}                % 粘贴源代码
\usepackage{xcolor}
\usepackage{color}
\definecolor{dkgreen}{rgb}{0,0.6,0}
\definecolor{gray}{rgb}{0.5,0.5,0.5}
\definecolor{mauve}{rgb}{0.58,0,0.82}
\usepackage{xcolor}

\definecolor{dkgreen}{rgb}{0,0.6,0}
\definecolor{gray}{rgb}{0.5,0.5,0.5}
\definecolor{comment}{rgb}{0.56,0.64,0.68}
\lstset{
  numbers=left,
  stepnumber=1,
  numberstyle=\tiny,
  numberstyle=\color[RGB]{0,192,192},
  frame=tb,
  aboveskip=3mm,
  belowskip=3mm,
  xleftmargin=2em,
  xrightmargin=2em,
  showstringspaces=false,
  keepspaces=true,
 columns=flexible,
  framerule=1pt,
  rulecolor=\color{gray!35},
  backgroundcolor=\color{gray!5},
  basicstyle={\small\ttfamily},
  numberstyle=\tiny\color{gray},
  keywordstyle=\color{blue},
  commentstyle=\color{comment},
  stringstyle=\color{dkgreen},
  breaklines=true,
  breakatwhitespace=true,
  tabsize=2,
}
% \lstset{
% breaklines=true,
%  %行号
%    numbers=left,
%     numberstyle= \tiny, 
%    basicstyle=\tiny,
%    %背景框
%    framexleftmargin=8mm,
%    frame=none,
%     %背景色
%    %backgroundcolor=\color[rgb]{1,1,0.76},
%     backgroundcolor=\color[RGB]{245,245,244},
%     %样式
%   keywordstyle=\bf\color{blue},
%   identifierstyle=\bf,
%    numberstyle=\color[RGB]{0,192,192},
%    commentstyle=\it\color[RGB]{0,96,96},
%   stringstyle=\rmfamily\slshape\color[RGB]{128,0,0},
%   %显示空格
%    showstringspaces=false
% }


%--------------------
\hypersetup{hidelinks}
\usepackage{booktabs}  
\usepackage{shorttoc}
\usepackage{tabu,tikz}
\usepackage{float}

\usepackage{multirow}



\tabcolsep=1ex
\tabulinesep=\tabcolsep
\newlength\tikzboxwidth
\newlength\tikzboxheight
\newcommand\tikzbox[1]{%
        \settowidth\tikzboxwidth{#1}%
        \settoheight\tikzboxheight{#1}%
        \begin{tikzpicture}
        \path[use as bounding box]
                (-0.5\tikzboxwidth,-0.5\tikzboxheight)rectangle
                (0.5\tikzboxwidth,0.5\tikzboxheight);
        \node[inner sep=\tabcolsep+0.5\arrayrulewidth,line width=0.5mm,draw=black]
                at(0,0){#1};
        \end{tikzpicture}%
        }

\makeatletter
\def\hlinew#1{%
  \noalign{\ifnum0=`}\fi\hrule \@height #1 \futurelet
   \reserved@a\@xhline}
   
\newcommand{\tabincell}[2]{\begin{tabular}{@{}#1@{}}#2\end{tabular}}%

\usepackage{subfigure}

\usepackage{CJK}
\usepackage{ifthen}


\usepackage{graphicx} 
\newcommand{\HRule}{\rule{\linewidth}{0.5mm}}

\newtheorem{Theorem}{定理}
\newtheorem{Lemma}{引理} 
%%使得公式随章节自动编号
\makeatletter
\@addtoreset{equation}{section}
\makeatother
\renewcommand{\theequation}{\arabic{section}.\arabic{equation}}

%-------------------------
	
\usepackage{pythonhighlight}
\usepackage{tikz}                    
\usepackage{tikz-3dplot}
\usetikzlibrary{shapes,arrows,positioning}

\bibliographystyle{gbt7714-2005}     %论文引用格式
%===================  定理类环境定义 ===================
\newtheorem{example}{例}              % 整体编号
\newtheorem{algorithm}{算法}
\newtheorem{theorem}{定理}            % 按 section 编号
\newtheorem{definition}{定义}
\newtheorem{axiom}{公理}
\newtheorem{property}{性质}
\newtheorem{proposition}{命题}
\newtheorem{lemma}{引理}
\newtheorem{corollary}{推论}
\newtheorem{remark}{注解}
\newtheorem{condition}{条件}
\newtheorem{conclusion}{结论}
\newtheorem{assumption}{假设}
%==================重定义 ===================
\renewcommand{\contentsname}{Contents}     
\renewcommand{\abstractname}{Abstract} 
\renewcommand{\refname}{Bibliography}     
\renewcommand{\indexname}{索引}
\renewcommand{\figurename}{Fig}
\renewcommand{\tablename}{Table}
\renewcommand{\appendixname}{附录}
\renewcommand{\proofname}{证明}
\renewcommand{\algorithm}{算法} 
