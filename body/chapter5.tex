\section{加速设备}
\subsection{fpga加速训练过程}
\subsubsection{opencl框架}\label{sec:opencl-step}
如\ref{sec:opencl}节所介绍的,要实现opencl设备加速,要从宿主机和目标机两方面分别入手.尤其是对于fpga来说,不能够做到实时编译,需要采用离线模式.
\subsubsection{如何配置fpga}
最终目标是要把需要fpga做的工作用opencl语言表达,然后通过intel提供的交叉编译工具编译成aocx文件,烧录到fpga中.具体的操作流程可以参考\cite{intel-opencl}其关键步骤如下:
\begin{enumerate}
  \item{\textbf{install board}} 将成c5p配套资源和sdk仪器安装到宿主机,并且配置好环境变量以及相关权限
  \item{\textbf{compile kernel}}  将opencl文件(里面包含了所有kernel函数的信息)编译成aocx镜像文件
  \item{\textbf{flash}} 利用编译好的aocx文件,将自启动镜像烧录到fpga中.通过重启fpga,使其自动执行镜像进入opencl模式.在该模式下,PCIE端口能够被宿主机所检测到
  \item{\textbf{program}} 将aocx中的所有信息烧录进fpga中,完成fpga的配置
\end{enumerate}

我们将其中的一些常用命令进行了封装,托管在我们的github\cite{rcClub}可以提高开发的效率.

通过intel 提供的sdk,我们可以很方便的完成opencl文件的烧录,核心在于如何编写cl文件
\subsubsection{cltorch}
cltorch是torch的一个发行版,其在支持大部分torch的功能,并且提供对opencl设备的支持.其本是为了支持opencl的gpu而设计的,但是基于它提供的cl文件,我们也可以实现对fpga的支持.具体步骤如下:
\begin{enumerate}
  \item 根据cltorch提供的cl文件,参照\ref{sec:opencl-step}节的过程,配置好fpga
  \item 将相关的库文件换成intel sdk提供的库文件
  \item 利用setDevice函数配置加速设备,并且将dtype设置为"torch.ClTensor"
\end{enumerate}
之后的代码就按照原先的写法既可.
到目前为止,我们配置好fpga后,还不能够被cltorch检测到,问题应该是出在替换为intel库文件这一步.我们在咨询技术部的时候,也知道intel的opencl相对于通用的opencl有一定的局限性.我们寄希望与在评审前将这一环节打通.
\subsection{movidius加速前向传播过程}
\subsubsection{修改源代码}

如\ref{sec:movidius}节所示,要实现movidius设备加速,需要将模型编译成需要的格式.
但是因为只是让movidius做推理部分的工作,所以需要对原先的评估代码做修改,只保留它的推理部分代码.具体修要修改的工作如下:
\begin{enumerate}
	\item 删除dropout层
	\item 删除数据导入读取部分个代码
	 \item	删除计算损失函数部分的代码
	 \item	删除除了输入层之外的所有placeholder
\end{enumerate}
然后添加Saver类,导出一个meta文件既可.

本项目修改后的版本托管在我们github\cite{my-fast-neural-sytle-tensorflow}上
\subsubsection{模型编译}
首先准备训练好的tensorflow模型,利用movidius自带的工具mvNCCheck,将tensorflow模型中的 .meta文件转换成ncs适用的graph文件
通过如下命令可以实现:
\begin{lstlisting}
mvNCCompile wave.meta -s 12 -in input -on output wave.graph
\end{lstlisting}
其中-in 和-on分别指定了输入层和输出层所对应的placeholder,需要和代码中的名称匹配
\subsubsection{顶层通信文件}
顶层通信代码主要包含以下几点:
1. 寻找ncs设备:
\addcode[python]{ncs-1.py}
2. 将之前转换好的graph文件导入到ncs设备中,为接下来的神经网络运行做好准备:
\addcode[python]{ncs-2.py}
3. 处理图片,将输入的图片处理成tensorflow需要的格式:
\addcode[python]{ncs-3.py}
4. 将处理好的图片发送到ncs中,获取输出的结果
\addcode[python]{ncs-4.py}
