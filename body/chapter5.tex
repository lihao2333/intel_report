\section{加速设备}
\subsection{fpga加速训练过程}
\subsubsection{opencl框架}\label{sec:opencl-step}
如\ref{sec:opencl}节所介绍的,要实现opencl设备加速,要从宿主机和目标机两方面分别入手.尤其是对于fpga来说,不能够做到实时编译,需要采用离线模式.
\subsubsection{如何配置fpga}
最终目标是要把需要fpga做的工作用opencl语言表达,然后通过intel提供的交叉编译工具编译成aocx文件,烧录到fpga中.具体的操作流程可以参考\cite{intel-opencl}其关键步骤如下:
\begin{enumerate}
  \item{\textbf{install board}} 将成c5p配套资源和sdk仪器安装到宿主机,并且配置好环境变量以及相关权限
  \item{\textbf{compile kernel}}  将opencl文件(里面包含了所有kernel函数的信息)编译成aocx镜像文件
  \item{\textbf{flash}} 利用编译好的aocx文件,将自启动镜像烧录到fpga中.通过重启fpga,使其自动执行镜像进入opencl模式.在该模式下,PCIE端口能够被宿主机所检测到
  \item{\textbf{program}} 将aocx中的所有信息烧录进fpga中,完成fpga的配置
\end{enumerate}

我们将其中的一些常用命令进行了封装,托管在我们的github\cite{rcClub}可以提高开发的效率.

通过intel 提供的sdk,我们可以很方便的完成opencl文件的烧录,核心在于如何编写cl文件
\subsubsection{cltorch}
cltorch是torch的一个发行版,其在支持大部分torch的功能,并且提供对opencl设备的支持.其本是为了支持opencl的gpu而设计的,但是基于它提供的cl文件,我们也可以实现对fpga的支持.具体步骤如下:
\begin{enumerate}
  \item 根据cltorch提供的cl文件,参照\ref{sec:opencl-step}节的过程,配置好fpga
  \item 将相关的库文件换成intel sdk提供的库文件
  \item 利用setDevice函数配置加速设备,并且将dtype设置为"torch.ClTensor"
\end{enumerate}
之后的代码就按照原先的写法既可.
到目前为止,我们配置好fpga后,还不能够被cltorch检测到,问题应该是出在替换为intel库文件这一步.我们在咨询技术部的时候,也知道intel的opencl相对于通用的opencl有一定的局限性.我们寄希望与在评审前将这一环节打通.
\subsection{movidius加速前向传播过程}
\subsubsection{movidius框架}
如\ref{sec:movidius}节所示,要实现movidius设备加速,要经过如下步骤.

首先准备训练好的tensorflow模型,利用movidius自带的工具mvNCCheck,将tensorflow模型中的 .meta文件转换成ncs适用的graph文件
通过如下命令可以实现:
\begin{lstlisting}
mvNCCompile wave.meta -s 12 -o wave.graph
\end{lstlisting}

其次,编写顶层的python文件,与ncs进行通信,将需要变换风格的图片输送到ncs中,再取回其运算结果,因为我们的项目是进行风格变换,所以返回的结果是一个多维向量,将其转换成图片输出就得到了我们需要的结果。
