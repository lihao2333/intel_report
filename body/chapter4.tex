\section{fast-neural-style 产生图片}
\subsection{核心原理}
其原理如图\ref{fig:fast-neural-style-principle}所示
\addfig[0.8]{fast-neural-style-principle.png}{fig:fast-neural-style-principle}{fast-nerual-style原理}
整个网络可以视为两部分,图形变换网络和失真网络.

图形变换网络是一个深度残余神经网络,由参数$W$表征.它将输入图像通过映射$\hat y=f_W(X)$转换为输出图像$\hat y$

每一个失真函数通过计算$l_i(\hat y,y_i)$来衡量输出图片$\hat y$和目标图片$y$的差异.

图形变换网络通过随机梯度下降法,以最小化总失真为目标进行训练.

目前采用的数据集是COCO dataset,里面包含了大量的通用性图片.

训练好模型之后,输入图像$x$经过一次前向传播,经过图形变换层就能够得到输出结果.所以速度非常快.
\subsection{代码结构}
\subsubsection{前向传播过程}
前向传播的过程比较清晰,只需要读取图片送入session中运行就可以,其中图片处理占的篇幅比较大,下面的代码是省略图片处理部分的代码.完整代码请看我们的github\cite{my-fast-neural-sytle-tensorflow}
\addcode[python]{eval.py}

\subsection{调用方式}

\subsubsection{API接口}
fast-neural-sytle提供的主要api以及默认值如表\ref{tbl-abi-neural-style}所示
\addtbl{api-fastneuralstyle}
\subsubsection{django调用fast-neural-style}
如图\ref{fig:top}所示,当接收到用户的post请求时,后台先获取内容图片和风格模型信息,存入数据库后调用fast-neural-style的api.在用户请求页面展示的时候,后台通过用户id查询数据库,将输出结果呈现.
产生图片的代码如下所示
\addcode[python]{gen-image.py}
其中第十行为通过系统命令调用fast-neural-style接口.其代码如下所示
\addcode[python]{func-fast-neural-style.py}
查看图片的代码如下所示
\addcode[python]{show-image.py}
