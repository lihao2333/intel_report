\section{Fast-neural-style produce pictures}
\subsection{mainly principle}
The principle is shown in figure\ref{fig:fast-neural-style-principle}所示
\addfig[0.8]{fast-neural-style-principle.png}{fig:fast-neural-style-principle}{Principle of fast-nerual-style}
The entire network can be viewed as two parts, a graphical transformation network and a distorted network.

The graphical transformation network is a deep residual neural network, characterized by the parameter $W$. It converts the input image by mapping $\hat y=f_W(X)$ to the output image $\hat y$

Each distortion function measures the difference between the output image $\hat y$ and the target image $y$ by calculating $l_i(\hat y,y_i)$.

The graph transformation network is trained by the stochastic gradient descent method with the goal of minimizing total distortion.

The data set currently used is the COCO dataset, which contains a large number of generic images.

After training the model, the input image $x$ is propagated once through the front, and after the graphics transformation layer can get the output. So it's highly speed.
\subsection{Code structure}

\subsection{Call method}

\subsubsection{API}
The main API provided by fast-neural-sytle and the default values are as follows\ref{tbl-abi-neural-style}
\addtbl{api-fastneuralstyle}
\subsubsection{django调用fast-neural-style}
如图\ref{fig:top}所示,当接收到用户的post请求时,后台先获取内容图片和风格模型信息,存入数据库后调用fast-neural-style的api.在用户请求页面展示的时候,后台通过用户id查询数据库,将输出结果呈现.
产生图片的代码如下所示
\addcode[python]{gen-image.py}
其中第十行为通过系统命令调用fast-neural-style接口.其代码如下所示
\addcode[python]{func-fast-neural-style.py}
查看图片的代码如下所示
\addcode[python]{show-image.py}